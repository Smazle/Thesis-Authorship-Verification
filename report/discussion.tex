\section{Discussion} \label{sec:discussion} 

\begin{itemize}

    \item

        Discuss that we would have liked to train an RNN on the character limit
        but that we did not have the computation power.

    \item

        Discuss how comparable our results will be with data in the real world.
        We had no texts written by actual ghost writers so we have no idea how
        our networks would perform on such texts.

    \item

        Discuss how different weight functions performed compared to other
        weight functions and describe why we think that is the case.

    \item

        Generate AUROC curves for our networks and discuss the output.

    \item

        Discuss that we beat our baselines but were not able to beat the MaCom's
        restrictions.

\end{itemize}


\subsection{Teacher Feedback}

In our introduction we reported that we wanted to look at what kind of feedback
we could give to teachers in conjunction with the bare predictions. As explained
earlier the system is not meant to be the final judge of which students are
cheating but are rather meant as a support system for teachers that are
already suspicious. We have looked at what kind of feedback we could give to
teachers. We have focused on the \gls{conv-char-NN} network since it performed
best on the test dataset. We have previously looked at the output of the
feature extraction layer to obtain information on what a specific network
were looking at. We wanted to do something similar for teacher feedback.
Recall that \gls{conv-char-NN} started with a convolutional layer followed
by a max pool layer. We therefore know that the larger the output of the
convolutional layer the more important that particular character sequence
is. The output of the feature extraction can be thought of as in Figure
\ref{fig:feature_extraction_output_example}. Each filter gives a single
output that is the maximum output in any filter position. The combining function
for the third network was the absolute difference. That means that when we
are comparing $t$ and $t'$. That means that the output of the combination
will be high for a particular filter iff the maximum output of that filter is
significantly different for $t$ and $t'$.

\begin{figure}
    \centering
    \textbf{Teacher Feedback Script Example}\par\medskip
    \includegraphics[width=\textwidth]{./pictures/discussion/teacher_feedback_example.png}
    \caption{Illustrates our script that gives feedback to teachers. The
        particular network used in this example only has three filters. The
        three filters maximum activations are shown in three different colors
        for the two texts they are comparing. The first filter looks for
        negative qualifiers. Therefore it reacts strongly to both the Danish
        word "ikke" (not) and the Danish word "ngen" (noone). The second filter
        looks for city names so it reacts strongly to the string "Rom " (Rome)
        but less strongly to "Der " (not a city name) even though it looks like
        a city name. The third filter reacts to phrases that contains the word
        "p\aa " (on) and therefore reacts about the same to both texts.}
    \label{fig:feature_extraction_output_example}
\end{figure}

It is hard to know exactly what the following layers does with the absolute
difference but we feel that it is a fair assumption that the largest filter
differences translates to the most important differences. The feedback system
we implemented for teachers takes an author $\alpha$, text $t$ and $n \in
\mathbb{N}^+$ and outputs the $n$ largest differences between each $t' \in
T_\alpha$ and $t$. The idea is that the when our system reports a negative the
teacher can ask for feedback from the system. The teacher will then get a list
of the $n$ greatest differences between each of the texts and can use that
information to argue against the student.

As an example we ran our system on a random author and a text that that author
did not write. The whole output of three different texts can be seen in Appendix
\ref{subsec:teacher_feedback_text_comparisons} through \ref{TODO}. We have shown
a truncated output in Table \ref{tab:teacher_feedback_output}.

\begin{table}
    \begin{tabular}{llll}
        \textbf{Filter} & \textbf{Activation Text 1} &
        \textbf{Activation Text 2} & \textbf{Difference} \\
        \hline
        426 & \verb'"; ”Hvis "'  & \verb'". Et kla"' & $|2.82 - 4.57| = 1.75$ \\
        71  & \verb'"dem; Nia"'  & \verb'"der skri"' & $|4.64 - 3.16| = 1.45$ \\
        549 & \verb'". Jeg vi"'  & \verb'". Her is"' & $|2.61 - 4.02| = 1.41$ \\
        288 & \verb'" 2 af 2\n"' & \verb'"og refle"' & $|4.90 - 3.53| = 1.37$ \\
        496 & \verb'" udseend"'  & \verb'"vad litt"' & $|2.78 - 4.12| = 1.34$ \\
        33  & \verb'"et sind."'  & \verb'"risning,"' & $|3.05 - 4.36| = 1.31$ \\
        460 & \verb'" 2 af 2\n"' & \verb'" krig og"' & $|3.69 - 2.43| = 1.26$ \\
        514 & \verb'", derfor"'  & \verb'", i forb"' & $|3.82 - 5.06| = 1.24$ \\
        531 & \verb'" om, for"'  & \verb'" om, hva"' & $|4.64 - 5.84| = 1.20$ \\
        458 & \verb'"Derudove"'  & \verb'"derfor b"' & $|4.31 - 3.13| = 1.18$ \\
        \\
        261 & \verb'".\n\n\n"'   & \verb'"– de"'     & $|1.72 - 2.70| = 0.98$ \\
        484 & \verb'"lv; "'      & \verb'" dét"'     & $|2.02 - 2.96| = 0.94$ \\
        17  & \verb'"st ”"'      & \verb'"gt; "'     & $|2.57 - 3.45| = 0.88$ \\
        145 & \verb'"ndet"'      & \verb'" Det"'     & $|2.37 - 3.25| = 0.88$ \\
        299 & \verb'"osse"'      & \verb'"– de"'     & $|3.12 - 3.98| = 0.86$ \\
        477 & \verb'"f 2\n"'     & \verb'"v og"'     & $|3.05 - 2.20| = 0.85$ \\
        421 & \verb'" 13."'      & \verb'" vel"'     & $|3.03 - 2.19| = 0.84$ \\
        434 & \verb'"am; "'      & \verb'" dét"'     & $|1.18 - 2.00| = 0.82$ \\
        27  & \verb'"n to"'      & \verb'"Dett"'     & $|2.20 - 2.99| = 0.79$ \\
        445 & \verb'" om,"'      & \verb'" – S"'     & $|2.51 - 3.27| = 0.76$ \\
    \end{tabular}
    \caption{Shows the 10 most different activations of convolutional filters on
        two different texts. Both the 10 most different activations for the
        filter of size 8 and size 4 is shown. The strings that produced the
        activation are shown and the actual activations are shown to the right.}
    \label{tab:teacher_feedback_output}
\end{table}

We also looked at whether or not we could say something about which parts of an
assignment are the most likely to be ghost written. That would allow a teacher
that is suspicious of a student to identify the part of the assignment that were
least likely to be written by him and look closely at that. To do that we wrote
a script that splits a text into paragraphs and use \gls{conv-char-NN} to get a
score for each paragraph. We can then report which of the paragraphs of the text
are the most likely to be ghost written. To test our script we chose an author
$\alpha$ from the validation dataset C with $|T_\alpha| = 19$. We also chose a
random text $t \in \overline{T_\alpha}$. We took out one of the texts $t' \in
T_\alpha$ and let $T = T_\alpha \setminus \{t'\}$. We then combined $t$ and $t'$
into $t''$ such that $t''$ consist of all of the text $t'$ followed by a random
paragraph from the text $t$. That is we constructed a text $t''$ that consisted
of a text in $T_\alpha$ followed by a single paragraph not from $T_\alpha$. We
then used our script to predict which part of $t''$ was most likely written by
a ghost writer. We would expect that the last paragraph in $t''$ would be the
most likely. The output of the script was a vector of the probability of each
paragraph having been written by a ghost writer.

\begin{equation}
    (0.66320, 0.60432, 0.54737, 0.47438, 0.30528)^T
\end{equation}

As expected the last paragraph is the least similar and therefore most likely to
be written by someone else. We are now able to give a teacher an idea of why our
network makes the predictions it does. We can give feedback on which parts of
the text is the least characteristic of an authors writing style and we can give
feedback on why the network thinks that. Of course a teacher will still have to
make the ultimate decision but can use the output of the network to underpin his
or her accusation.
