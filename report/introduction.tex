\section{Introduction} \label{sec:introduction}

% An introduction to the context or background of the topic (you could include
% interesting facts or quotations)

In this thesis we work on the problem of authorship verification using texts
written by Danish secondary school pupils. Authorship verification and
authorship attribution, is the ability to distinguish between authors of texts,
based on a set of extracted textual features. The automation of authorship
attribution/verification has been a lively branch of research ever since the
beginning of the digital age, giving birth to online digital text forensics
tasks, such as \cite{pan:2015}. Initial attempts at quantifying writing style
can be seen by \cite{Mendenhall237}, who attempted to determine the authorship
of several of Shakespeare's texts. There is a theory that Shakespeare did not
write some or all of his texts, or that he was at least a front for one or more
unknown authors. \cite{Mendenhall237} attempted this classification, using the
frequency distribution of words of different lengths. Throughout the years the
approaches to this problem has changed quite a bit. When authorship attribution
started to interest researchers the approaches were Stylometric. In addition to
that, fully automated systems were rare as authorship attribution/verification
was mostly used in an supporting manner. It was during the 1990's that fully
automated systems became more prevalent. The main reason for this was the
Internet. Before the Internet, the data available simply was not suitable for
authorship attribution tasks. Books were too big, resulting in a lack in
homogeneity, and the amount of authors, and bench-marking data was to small.
The Internet paved the way for insurmountable amount of data, and variations of
that data, impacting areas such as information retrieval, machine learning and
\gls{NLP}.

In order for any fully automatic authorship verification to work, stylometric
features describing the text has to be automatically extracted. These features
span multiple linguistic layers, ranging from the low level character n-grams,
to the high level application specific features such as text creation date,
and number of edits. It is using these features many of the current day
state-of-the-art approaches are based.

% The reason for writing about this topic:

In this thesis we want to experiment with and solve an authorship
verification task for the Danish company \texttt{MaCom A/S}
\footnote{\url{http://www.macom.dk/}}. MaCom is the company behind the product
\texttt{Lectio} \footnote{\url{https://www.lectio.dk/}}, which is a website
that allows for student administration, communication, and digital teaching
aid. Lectio is used in schools all over Denmark. A service the website offers,
is the submission and handling of assignments written by students throughout
their enrollment. MaCom has shown interest in determining whether or not these
assignment were possible written by someone other than the student (a "ghost
writer"). Ghost writing is especially a problem on the \gls{SRP} assignment.
\gls{SRP} is an interdisciplinary assignment all Danish secondary school
students turn in at the end of their third year. There is no oral examination
for the assignment and the grade obtained is part of the students final results
from the secondary school. The combination of the importance of the assignment
and no oral examination leads to students turning in assignments written by
ghost writers. The Danish state owned public service radio and television
company \texttt{DR} has written an article describing the ghost writer problem
\footnote{\url{https://www.dr.dk/nyheder/indland/elever-bruger-ghostwritere-til-
eksamen}}. The article describes that when asking 2000 student, 58\% got help
from friends or family, and around 15\% knew someone who had their assignment
written by someone else. In this thesis we setup a system for detecting ghost
writing based on machine learning methods. The system is meant to help teachers
make decisions about whether or not an assignment turned in by a student is
written by someone else. It is not important that the system catches 100\% of
the assignments written by someone else. If the system only catches a fraction
of the cheaters it will function as a deterrent for other students cheating.
What is most important is that the system does not accuse anyone of cheating
who has turned in their own assignment. The system should also be able to give
a reason for why we think a particular assignment is written by someone else.
Such a reason could for example be that the frequency of particular words are
significantly different in the new assignment than in all previously handed in
by the student. Reasons for why we think an assignment is written by someone
else will help a teacher if he/she wants to accuse a student of using a ghost
writer.

% Introduce the main ideas that stem from your topic/title and the order in
% which you will discuss them?

A more formal definition of the problem we will be working with are,

\begin{definition}[Authorship Verification]

    Given a set of texts $T_\alpha$ written by author $\alpha$ and a single
    text $x$ of unknown authorship. Determine if $x \in T_\alpha$.

\end{definition}

Authorship verification is closely linked with the problem of authorship
attribution,

\begin{definition}[Authorship Attribution]

    Given a set of authors $\mathcal{A} = \{ \alpha_1, \alpha_2,...\alpha_N\}$,
    each with set of text $T_{\alpha_i}$, and a text of unknown authorship
    \texttt{x}. Determine which $\alpha_i \in \mathcal{A}$ is the author of
    \texttt{x}.

\end{definition}

The problems are closely linked since an answer for authorship attribution
can be obtained by using authorship verification and an answer for authorship
verification can be obtained by using authorship attribution. Consider a
case where we are given a solution to the authorship verification problem
$\mathcal{S}$. $\mathcal{S}$ is a mapping from an author $\alpha$ and text
$x$ to either true or false. Given an instance of the authorship attribution
problem with authors $\mathcal{A}$ and text $x$ we solve the problem by using
$\mathcal{S}$ on each author $\alpha \in \mathcal{A}$. We return the author
where $\mathcal{S}$ reports true. Now consider a case where we are given a
solution to the authorship attribution problem $\mathcal{S}$. $\mathcal{S}$ is
now a mapping from a set of authors $\mathcal{A}$ and text $x$ to an author
$\alpha \in \mathcal{A}$. Given an instance of the authorship verification
problem with author $\alpha_i \in \mathcal{A}$ and text $x$ and a set of
texts written by different authors $\overline{T}_{\alpha} = \bigcup\{T_\beta
| \beta \in \mathcal{A} \land \beta \neq \alpha\}$ we solve the verification
problem by applying the attribution function to the texts $T_{\alpha} \cup
\overline{T}_{\alpha}$ and the text of unknown authorship $x$. If $x \in
T_{\alpha}$ back we report true and otherwise false.
