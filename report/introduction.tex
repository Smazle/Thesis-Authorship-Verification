% Introduction should be written in the present tense!
%
% Your introduction needs to include background information which is generally
% accepted as fact in a discipline. You also need to explain why the research
% you are reporting is important. It is usually presented in the present tense.
%   - https://services.unimelb.edu.au/__data/assets/pdf_file/0009/471294/Using_tenses_in_scientific_writing_Update_051112.pdf

\section{Introduction} \label{sec:introduction}

% An introduction to the context or background of the topic (you could include
% interesting facts or quotations)

In this thesis we work on the problem of authorship verification using texts
written by Danish secondary school pupils (students from roughly age 14-18).
Authorship verification and authorship attribution describe the ability to
distinguish between authors of texts such that when given a text of unknown
authorship it can either be attributed to an author or verified to be written
by a specific author. The automation of authorship attribution/verification
has been an active branch of research ever since the beginning of the digital
age, giving birth to online digital text forensics tasks such as the work
by \citet{pan:2015}. Initial attempts at quantifying writing style can be
seen in \citet{Mendenhall237}, who attempted to determine the authorship of
several of Shakespeare's texts. A theory exists that Shakespeare did not write
all of his texts, or that Shakespeare was a pseudonym for one or several
unknown authors. \citet{Mendenhall237} attempted his classification using the
frequency distribution of words of different lengths. Throughout the years
the approaches to this problem have changed quite a bit. When authorship
attribution initially sparked an interest amongst researchers, the approaches
were \textit{Stylometric} i.e. they were based on the linguistic style of
authors. In addition, fully automated systems were rare and were mostly used
in a supporting manner. It was during the 1990's that fully automated systems
became more prevalent. The main reason for this was the internet. Before the
Internet, the data available simply was not suitable for authorship attribution
tasks. Books were too big resulting in a lack of homogeneity, and the amount
of authors and bench-marking data was too small. The internet paved the way
for insurmountable amounts of data and variations of that data, impacting
areas such as \textit{information retrieval}, \textit{machine learning} and
\gls{NLP}.\cite{stamatos2009}

In order for any fully automatic authorship verification/attribution to work,
stylometric features describing the text have to be automatically extracted.
These features can span multiple \textit{linguistic layers} ranging from low
level character sequence frequencies to the high level application specific
features such as text creation date and number of edits. Many of the current day
state-of-the-art approaches are based on these features \citep{stamatos2009}.

% The reason for writing about this topic:

We want to experiment with and solve an authorship verification task for the
Danish company MaCom \footnote{\url{http://www.macom.dk/}}. MaCom is the company
behind the product Lectio \footnote{\url{https://www.lectio.dk/}} which is
a website that provides student administration, communication, and digital
teaching aid. Lectio is used in more than 90 \% of Danish secondary schools
\footnote{\url{https://www.business.dk/digital/df-ordforer-om-pivabent-lectio-de
t-kan-vaere-farligt-hvis-man-taenker}}. One of the services the website offers
is the submission and handling of assignments written by students throughout
their school years. MaCom has shown interest in determining whether or not these
assignment are written by someone other than the student (a ``ghostwriter'').
Ghost writing is especially a problem on the \gls{SRP} assignment. The \gls{SRP}
is an interdisciplinary assignment all Danish secondary school students turn in
during their last year. There is no oral examination for the assignment and the
grade obtained is part of the student's final results from secondary school.
The combination of the importance of the assignment and no oral examination
leads to some students turning in assignments written by ghostwriters. The
Danish state owned public service radio and television company Danmarks Radio
(\texttt{DR}) has written an article describing the ghostwriter problem
\footnote{\url{https://www.dr.dk/nyheder/indland/elever-bruger-ghostwritere-til-
eksamen}}. The article states that out of 2000 students, 58\% got
help from friends or family, and around 15\% knew someone who had
their entire assignment written by someone else. Furthermore it has
been estimated that 4\% of turned in \glspl{SRP} are ghostwritten
\footnote{\url{https://politiken.dk/indland/uddannelse/art5603163/Gymnasieelever-\%C2\%BBSnyderi-beviser-hvor-vanvittig-betydningsfuld-SRP-er-blevet\%C2\%AB}}.

In this thesis we set up a system for detecting ghost writing using machine
learning methods. The system is meant to help teachers decide whether or not
an assignment is written by the student who turned it in or by someone else.
According to Macom, the number one priority is to catch cheaters while accusing
as few innocent students as possible. We should focus on minimizing the number
of falsely accused students even if it means that we catch fewer cheaters. As
long as the system reliably catches a few students, it will deter the other
students from cheating. The system should also be able to give evidence for why
we think a particular assignment is written by someone else. Such evidence could
for example be that the frequency of particular words is significantly different
in the new assignment compared to previous assignments by the student. This
product would work in a supplementary manner, providing input for teacher to
look over when making the final decision.

\subsection{Notation} \label{subsec:notation}

We work with several different objects. The main ones are texts and authors.
We generally name texts $t$ and authors $\alpha$. If a few texts or authors
are needed we generally call them $t, t', \dots$ and $\alpha, \alpha', \dots$
and if many are needed we call them $t_1, \dots, t_n$ and $\alpha_1, \dots,
\alpha_n$. The set $\mathcal{A} = \{\alpha_1, \alpha_2, \dots, \alpha_n\}$
denotes all authors, where $T_\alpha$ denotes the set of texts written by
the author $\alpha$. The set $\mathcal{T} = \bigcup_{\alpha \in \mathcal{A}}
T_\alpha$ subject to $T_{\alpha} \cap T_{\alpha'} = \emptyset$ for all $\alpha
\neq \alpha'$ denotes the set of all texts. The set $\overline{T_\alpha} =
\mathcal{T} \setminus T_\alpha$, denotes the texts which are not written by
$\alpha$.

Texts are represented by a sequence of characters. We denote the
length of that sequence $|t|$, for any text $t$. Each text also has a point
of time where it was written. This time is given by the function $\tau \colon
\mathcal{T} \rightarrow \mathbb{N}^+$, which returns the point in time where the
assignment was submitted in months relative to the earliest submitted $t \in
\mathcal{T_\alpha}$. Consider an author $\alpha$ that has written 3 assignments.
One in April 2018, one in March 2018 and one in December 2017. The function
$\tau$ would then return for each assignment respectively 0, 1 and 4.

For vectors we use the standard notation of naming them in lower case bold
letters as $\mathbf{x}$. Matrices will be named in upper case non bold
letters. We let elementwise multiplication of both vectors and matrices be
denoted with $\otimes$ and elementwise summation as $\oplus$. We will denote
submatrices as $X[a,b;c,d]$ which means the submatrix of $X$ that consist of
rows $a$ to $b$ and columns $c$ to $d$.

Much existing work on authorship verification is based on frequencies of
sequences of object, \textit{n-grams} \citep{stamatos2009}. For example, the
object could be characters and the n in n-gram would then refer to the length of
character sequences in a text. If we take the text to be "hello" the different
character-3-grams will be ``hel'', ``ell'' and ``llo''. The frequency of each
of them is $\frac{1}{3}$, and would serve as the input for a model. We will use
n-grams in some of our baseline methods and some of our networks will use them
indirectly.

% Introduce the main ideas that stem from your topic/title and the order in
% which you will discuss them?

As described the problem we try to solve in this thesis is authorship
verification which is defined below.

\begin{definition}[Authorship Verification]
    \label{def:authorship_verification}

    Given a set of texts $T_\alpha$ written by author $\alpha$ and a single text
    $t$ of unknown authorship, determine if $\alpha$ is the author of $t$.

\end{definition}

Authorship verification is closely linked with the problem of authorship
attribution as can be seen in the definition of authorship attribution shown
below.

\begin{definition}[Authorship Attribution]

    Given a universe of authors $A = \{\alpha_1, \alpha_2,\dots,\alpha_n\}$,
    each with a set of texts $T_{\alpha_i}$, and a text $t$ of unknown
    authorship, determine which $\alpha_i \in A$ is the author of $t$.

\end{definition}

The problems are closely linked since an answer for authorship attribution can
be obtained by using authorship verification, and an answer for authorship
verification can be obtained by using authorship attribution. Consider a
case where we are given an oracle $\mathcal{S}$ answering the authorship
verification problem. The oracle $\mathcal{S}$ is a mapping from an author
$\alpha$ and text $t$ to either true or false. Given an instance of the
authorship attribution problem with authors $A$ and text $t$ we solve the
problem by using $\mathcal{S}$ on each author $\alpha \in A$. We return
the author where $\mathcal{S}$ reports true. Now consider a case where we
are given a solution to the authorship attribution problem $\mathcal{S}'$.
The solution $\mathcal{S}'$ is now a mapping from a set of authors $A$ and
text $t$ to an author $\alpha \in A$. Given an instance of the authorship
verification problem with author $\alpha_i \in A$ and text $t$ and a set
of texts written by different authors $\overline{T}_{\alpha}$ we solve the
verification problem by applying $\mathcal{S}'$ to the texts $T_{\alpha} \cup
\overline{T}_{\alpha}$ and the text of unknown authorship $t$. We then return
$\mathbbm{1}\left[\mathcal{S}'(A, t) = \alpha\right]$.

\subsection{Roadmap}

In Section \ref{sec:related_work} we go through previous work on authorship
verification/attribution and previous work performed on MaCom's dataset. We
specifically focus on neural network methods as we want to expand on those
methods.

In Section \ref{sec:method} we describe the theory behind the methods we work
with. We start out by briefly describing our baseline methods after which we
describe the neural network methods. Specifically we describe the theoretical
foundations of \textit{Siamese neural networks} that we have focused on.

In Section \ref{sec:data} we describe the dataset we have been using throughout
our experiments. We present statistics about average number of texts and the
number of authors etc. We also describe the preprocessing steps we performed on
the dataset.

In Section \ref{sec:experiments} we go through the different network
architectures we used. We present the results we obtained on a validation
dataset and how we reached the final architectures we settled on.

In Section \ref{sec:results} we present our test set performances for our
baseline methods and neural network methods.

In Section \ref{sec:discussion} we discuss the results we presented in the
previous section. We discuss the applicability of the method we developed to the
real world and we discuss how the system can be used if implemented.

In Section \ref{sec:conclusion} we present our conclusion based on the results
section and discussion section.

In Section \ref{sec:future_work} we go through how the methods could be further
developed in the future and what we did not achieve in our project.
