\section{Related Work} \label{sec:related_work}
%The problem of authorship verification

Our work in this thesis is inspired by the previous work of several researchers.
\cite{DBLP:journals/corr/RuderGB16c} shows a Neural Network for authorship
attribution. Authorship attribution is closely connected to authorship
verification as every authorship attribution problem can be transformed into a
series of authorship verification problems. To attribute the author of a text
you can perform a series of authorship verifications of each candidate author
and return the author that reported true. Their experiment consisted of a
network where they first had a Convolutional layer, after that a max-over-time
pooling layer and then a densely connected network on the top of that. Character
level features has previously been shown to be important for both authorship
verification and attribution (TODO: cite something). The hope was that the
convolutional layer would learn important features from sequences of characters.
The max-over-time pooling would take the most important value from each
convolutional filter and would extract a similar number of features for each
text even though the texts are of differing length. The dense network was then
supposed to take the features extracted from the text and determine authorship
of the text from them.

\cite{DBLP:journals/corr/RuderGB16c} also used multiple channels in their
network. Each channel was a different token sequence some of them were word
embeddings and some were character embeddings. Some of the channels were static
while some of the channels were non-static meaning that the word/char-embedded
vectors would change during training. The point of the channels was that the
network were able to extract features from multiple levels of features (TODO:
reference some explanation of different levels of features). Specifically they
used networks with the following channels

\begin{description}
    \item[CNN-char:] Single non-static character channel.
    \item[CNN-word:] Single non-static word channel.
    \item[CNN-word-word:] Two word channels, one non-static and one static.
    \item[CNN-word-char:] Two non-static channels one for words and one for
        characters.
    \item[CNN-word-word-char:] One static word channel, one non-static word
        channel and one non-static character channel.
\end{description}

The best performing configuration was the CNN-char.

The method implemented by \cite{shrestha2017}, was their attempt at \gls{AA}
on short texts. The reasoning behind the only focusing on short texts was the
advent of social media, and the great usage of E-mail. Their approach makes use
of a \gls{CNN}. This \gls{CNN} only takes in a sequence of character-n-grams.
The reasoning for this usage of only char-n-grams was the small amount of text
in each sample. By passing these N-grams through a Embedding layer, a 25\%
dropout layer, 3 convolutional layers and then using max-over-time, they get
a compact representation of the text. They hypothesize this representation
captures the morphological, lexical and syntactic level of the supplied text.
This compact representation is then parsed through a fully connected soft-max
layer, to produce a probabilistic distribution over all authors. In order to
test their method they made use of a twitter data set, containing approximately
9000 user, all having over a 1000 tweets to their name. They made use of two
different configurations of their networks. One using character-1-grams and one
using character-2-grams. After removing bot-like authors, they got an accuracy
of 0.678, and 0.683 respectively. This however, was only with 35 authors used,
and 1000 tweets per author. In the case where either the authors count was
increased or the number of tweets was decreased, the accuracy quickly worsened.
In order to extract some sort of meaning from the predictions they made using
this approach, they made use of the saliency score to determine the impact each
n-gram had on the final decision.
