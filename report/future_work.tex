\section{Future Work} \label{sec:future_work}

% Try a weight function that use actual prediction scores to define time
% weights.

\begin{itemize}

    \item Try applying our methods on other datasets
    \item Writing style development over time i future work.

\end{itemize}

A concern brought up in Section \ref{sec:app_of_method}, regard the
applicability of our current methods to other courses, or even another type
of text data like code for example. Due to the speculatory nature of that
discussion further work within this subject, could well include experimenting
with applying our network design to other types of data. This could include
other go from other high-school classes, such as math or history, to other
educational level, such as elementary school and university.

Another concern brought in Section \ref{sec:app_of_method}, was the inclusion of
citation in the texts provided to our models. Citation are mostly a source of
noise for our models. This makes the most sensible solution, when working with
these models in future, to simply remove the citations before training and
applying the methods. 

There is also the scenario where a group of students do a collective handin of a
collaborative assignment. As mentioned in Section \ref{subsubsec:group_handin}
this is not something that our would be able to handle out of the box. However
a future modification of our methods could tackle this, using a paragraph based
approach. \ref{subsubsec:ghost_written_areas} One could loop through each
paragraph in the collaborative text, and then apply our methods. The paragraph
would be tested against all the students who worked on the assignment. The goal
would be to attribute each paragraph to one of the proposed writers of the
assignment. If a certain amount of paragraphs were unattributable, then the
entire text would be considered the product of a ghost writer. 
In this scenario a subset of student in the group might have had their parts
produced by a third party ghost writer, and some of them did not.

