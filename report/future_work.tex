\section{Future Work} \label{sec:future_work}

A concern brought up in Section \ref{subsec:applicability_of_method} regard
the applicability of our current methods to other courses or other types of
text data like code. It would be interesting to apply our methods to texts from
other secondary school courses and observe the performance. It would also be
interesting to apply the methods to other age classes such as primary school
students or college students. Application to primary school students might be
hard since assignments are typically shorter and writing style might change even
more than during secondary school. However application to college students might
lead to very encouraging results since writing style probably remains more fixed
and assignments are generally longer.

Another concern brought in Section \ref{subsec:applicability_of_method}, was
the inclusion of citation in the texts provided to our models. Citation are
mostly a source of noise for our models. This makes the most sensible solution
when working with these models in future to simply remove the citations before
training and applying the methods. Doing so successfully might increase the
accuracy of our models further.

There is also the scenario where a group of students do a collective handin of a
collaborative assignment. As mentioned in Section \ref{subsubsec:group_handin}
this is not something that our methods would be able to handle out of the box.
However a future modification of our methods could tackle this using a paragraph
based approach. As described in Section \ref{subsubsec:ghost_written_areas} one
could loop through each paragraph in the collaborative text, and then apply
our methods. The paragraph would be tested against all the students who worked
on the assignment. The goal would be to attribute each paragraph to one of the
proposed writers of the assignment. If a certain amount of paragraphs were
unattributable then the entire text would be considered the product of a ghost
writer. In this scenario a subset of student in the group might have had their
parts produced by a third party ghost writer.

A big focus during our experiments was keeping the model generalizing, and
not author specific. Future work could include verifying the reach of this
generalizability. Our primary concern was the generalizability with regard to
secondary school students, but what if the generalizability extends to a larger
group of authors than that. This would mean the model was more powerful than
originally thought, thus expanding the areas of usability quite a bit. Tests
of this could include applying the our models to another dataset. This could
be anything from scientific papers to newspaper articles, anything where the
author has a substantial amount of work behind him.

As we briefly mentioned in Section \ref{subsec:writing_style_changes} students
writing style can be seen to change over time. It would be interesting to study
exactly what changes occur. That might lead to insights into teaching methods.
We could also look at finding students with little writing style development and
flagging them as needing extra help from teachers.
