\begin{abstract} \label{sec:abstract}

    In this thesis we investigate authorship verification of texts produced by
    secondary school students. Authorship verification or \textit{ghost writer
    detection} is the process of determining whether a text is written by an
    author given a set of texts written by the same author. We work with the
    Danish company MaCom that provides a dataset containing assignments from
    secondary school students. We focus on deep neural networks to perform the
    authorship verification. We implement two baseline methods representing
    classic machine learning solutions to the authorship verification problem.
    After that we present three networks to solve the same problem:

    \begin{itemize}

        \item

            A convolutional neural network working on the character level of the
            texts,

        \item

            A recurrent neural network working on the sentence level of the
            texts,

        \item

            A convolutional neural network working on the character and word
            level of the texts.

    \end{itemize}

    Classic machine learning methods for authorship verification use manually
    chosen feature configurations. The networks we implement extract features
    from raw text data. The networks beat both of the baseline methods on all
    relevant parameters. On a dataset with 50\% ghost written assignments we
    achieve an accuracy of 86.5\%.

    Our methods are meant to be used in a supporting manner for teachers of
    secondary schools for detecting ghost written assignments. We are able to
    give teachers feedback on why the networks makes a decision and we are able
    to detect specific areas of assignments that might be ghost written.

\end{abstract}
