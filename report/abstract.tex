\begin{abstract} \label{sec:abstract}

    In this thesis we investigate authorship verification of texts produced
    by secondary school students. Given a set of texts written by one author,
    authorship verification (or \textit{ghostwriter} detection) is the process
    of determining whether a text of unknown authorship is written by said
    author. We work with the Danish company MaCom that provides a dataset
    containing assignments from secondary school students. We focus on deep
    neural networks to perform the authorship verification. We implement two
    baseline methods representing classic machine learning solutions to the
    authorship verification problem. After that we present three networks to
    solve the same problem:

    \begin{itemize}

        \item

            A convolutional neural network working on the character level of the
            texts,

        \item

            A recurrent neural network working on the sentence level of the
            texts, and

        \item

            A convolutional neural network working on both the character and
            the word level of the texts.

    \end{itemize}

    Classic machine learning methods for authorship verification use manually
    chosen feature configurations, but the networks we implement extract
    features from raw text data. The networks beat both baseline methods
    on accuracy and accusation error. On a dataset with 50\% ghostwritten
    assignments we achieve an accuracy of 86.5\%.

    Our methods are meant to be used by teachers of secondary schools in a
    supporting manner to detect ghostwritten assignments. They are able to give
    teachers feedback on why the networks make a decision and they are able to
    detect specific areas of assignments that might be ghostwritten.

\end{abstract}
