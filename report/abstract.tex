\begin{abstract} \label{sec:abstract}

    In this report we investigate different methods for authorship verification.
    Authorship verification is the process of determining whether a text is
    written by an author given a set of texts written by the same author. We
    will implement a select few of the algorithms we investigate. The specific
    algorithms are:

    First, the Delta Method which we use as a baseline for the other methods we
    implement. The Delta Method is a distance based approach that use features
    (vector of numbers) extracted from the texts and a distance metric to find
    the closest author of an unknown text.

    Second, a generalising Random Forest approach is implemented. The method
    encodes features extracted from the texts using a \gls{UBM} and applies the
    Random Forest to those encoded features.

    Third, the Delta Method is expanded by trying different features and
    distance metrics than used in the original Delta Method.

    Fourth, an Author Specific \gls{SVM} is implemented. The SVM is trained
    using the imposter method. Two sets of texts are generated. One set of texts
    known to be written by the author and another set of texts known to not be
    written by the author. Features are then extracted from all texts and a
    \gls{SVM} is then trained on those sets of features.

    For training and evaluation we use two datasets. The datasets are from two
    instances of a yearly competition in text forensics (PAN). Specifically we
    use the dataset from the 2013 edition and the 2015 edition. We obtain the
    third best result on the PAN 2013 task and the eights best result on the PAN
    2015 task.

\end{abstract}
