\section{Method} \label{sec:method} 

TODO: Describe how our first architecture was inspired by \cite{qian:2018} but
instead of using the cosine difference on top of the NN extracting features, we
used a dense neural network to compare the output.

\begin{itemize}
    \item Replace escaped characters in MaCom data.
    \item Convert all characters in text to numbers such that we have lists of
        numbers.
    \item Convert list of numbers to list of one hot encoded vector where each
        character has a different vector.
    \item Pad all texts with zeroes such that all texts has the same length.
    \item Make random dataset of size n by making (n/2) rows containing two
        padded texts from the same author and the class 1 and making (n/2) rows
        containing two texts from different authors and the class 0.
    \item Feed the problems to a neural network that looks like this,

        \begin{lstlisting}
     |Text 1|          |Text 2|
        |                 |
        |                 |
      | Convolutional Layer(s) |
          |                 |
          |                 |
  | GlobalMaxPool | | GlobalMaxPool |
          |                 |
          |                 |

    |     Dense Layer(s)        |
                |
                |
| Same/Different Author Probabilities |
        \end{lstlisting}

        The number of filters in the convolutional layers determines the number
        of outputs of the GlobalMaxPools. The output of the GlobalMaxPools
        represents the input text and the dense layers then have to compare the
        two representations to find whether or not they are written by the same
        author.

        The convolutional layers look at some number of characters at a time and
        the hope is then that they will learn the important n-grams.

    \item Try the delta method and our SVM method as a baseline for the neural
        networks.

\end{itemize}

TODO:

\begin{itemize}
    \item Describe preprocessing.
\end{itemize}
