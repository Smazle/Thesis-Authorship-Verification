\documentclass[10pt]{beamer}

\usetheme[progressbar=frametitle]{metropolis}

\usepackage{amsmath}
\usepackage{amsthm}
\usepackage{amssymb}
\usepackage{appendixnumberbeamer}
\usepackage{booktabs}
\usepackage[scale=2]{ccicons}
\usepackage{pgfplots}
\usepgfplotslibrary{dateplot}
\usepackage{xspace}
\usepackage{graphicx}
\usepackage{subcaption}
\usepackage{bbm}

\newcommand{\themename}{\textbf{\textsc{metropolis}}\xspace}

\title{Authorship Verification}
\subtitle{Masters Thesis}
\date{}
\author{August S\o rensen \& Magnus Stavngaard}
\institute{University of Copenhagen}

\defbeamertemplate{description item}{align left}{\insertdescriptionitem\hfill}

\begin{document}

\maketitle

\begin{frame}[fragile]{Problem Statement}
    \begin{itemize}
        \item A system to identify ghostwritten texts.
        \item The system should accuse as few innocent students as possible.
        \item The system should catch as many ghostwriters as possible.
        \item The system should be able to give feedback to teachers.
        \item Less than 10\% accusation error and more than 95\% specificity.
    \end{itemize}
\end{frame}

\begin{frame}[fragile]{Solution Architecture}
    \begin{center}
        \includegraphics[width=0.45\textwidth]{../../macom/summary/pictures/Model}
    \end{center}
\end{frame}

\begin{frame}[fragile]{Siamese Networks}
    \begin{itemize}
        \item Siamese Neural Networks compares two objects.

            \begin{description}
                \item[Input] Two objects,
                \item[Output] Probability that objects belong to same class.
            \end{description}
    \end{itemize}

    \begin{center}
        \includegraphics[width=0.6\textwidth]{../../report/pictures/method/siamese}
    \end{center}
\end{frame}

\begin{frame}[fragile]{Networks}
    \setbeamertemplate{description item}[align left]
    \begin{itemize}
        \item Networks consist of 4 parts:

            \begin{description}
                \item[Embedding] Encode raw texts in format suiting networks.
                \item[Feature Extraction] Extract feature vectors from the
                    encoded texts.
                \item[Combining] Combine extracted feature vectors using some
                    function.
                \item[Decision] Decide the probability that two texts are from
                    the same author.
            \end{description}
    \end{itemize}
\end{frame}

\begin{frame}[fragile]{Char-CNN}
    \begin{center}
        \includegraphics[width=0.6\textwidth]{../../macom/summary/pictures/model}
    \end{center}
\end{frame}

\begin{frame}[fragile]{Combining Network Output}
    \begin{itemize}
        \item Networks compare two texts at a time but an author has several
            texts.
        \item We combine the output of the network using a weighted average.
        \item Recent assignments are weighted as more important than older
            assignments.
        \item Longer assignments are weighted higher than shorter assignments.
    \end{itemize}
\end{frame}

\begin{frame}[fragile]{Results}
    \begin{itemize}
        \item Evaluated on a dataset of 50\% negatives and a dataset with 4\%
            negatives.

            \begin{center}
                \begin{tabular}{l|ll}
                                              & \textbf{4\%} & \textbf{50\%} \\
                    \hline
                    \textbf{Accusation Error} & 23.52\%      & 9.87\%        \\
                    \textbf{Specificicity}    & 8.50\%       & 82.06\%       \\
                    \textbf{Accuracy}         & 96.11\%      & 86.53\%       \\
                \end{tabular}
            \end{center}
    \end{itemize}
\end{frame}

\begin{frame}[fragile]{Teacher Feedback}
    \begin{center}
        \includegraphics[width=\textwidth]{../../report/pictures/discussion/teacher_feedback_example}
    \end{center}
\end{frame}

\begin{frame}[fragile]{Teacher Feedback}
    \begin{center}
        \scriptsize
        \begin{tabular}{lll|lll}
            \textbf{Max 1}    & \textbf{Max 2}    & \textbf{Max 3}       &
            \textbf{Text 1}   & \textbf{Text 2}   & \textbf{Difference}  \\
            \hline
            \verb[nemlig 1[   & \verb[nemlig 1[   & \verb[nemlig –[      &
            \verb'pere.\n\nH' & \verb'nemlig b'   & |3.06 - 4.70| = 1.64 \\

            \verb[, F.eks.[   & \verb[, F.eks.[   & \verb[, F.eks.[      &
            \verb'. F.eks.'   & \verb'del. Und'   & |4.73 - 3.18| = 1.55 \\

            \verb[ke …''. [   & \verb[a...''\n\n[ & \verb[n''. '' [      &
            \verb'v. Men d'   & \verb'n. Den v'   & |4.28 - 2.91| = 1.37 \\

            \verb[forsøger[   & \verb[forsøger[   & \verb[forsøger[      &
            \verb'for sætt'   & \verb'forsøgte'   & |3.40 - 4.77| = 1.37 \\

            \verb[, Hvorda[   & \verb[, Hvorda[   & \verb[,08 – 6,[      &
            \verb', som ti'   & \verb', Hvorda'   & |3.70 - 5.07| = 1.37 \\

            \verb[der; ’’M[   & \verb[der; ”Ha[   & \verb[der; ”Ma[      &
            \verb'dem; Nia'   & \verb'der omha'   & |4.64 - 3.28| = 1.36 \\

            \verb[. Her ef[   & \verb[. Her ef[   & \verb[' Her br[      &
            \verb'. Jeg vi'   & \verb'. Her fo'   & |2.61 - 3.92| = 1.31 \\

            \verb[r dog kr[   & \verb[r dog kr[   & \verb[r dog ’d[      &
            \verb'r og lud'   & \verb'r dog i '   & |2.83 - 4.13| = 1.30 \\

            \verb[11], da [   & \verb[:1], da [   & \verb[:1], da [      &
            \verb'ys”, der'   & \verb'for, da '   & |3.78 - 5.04| = 1.26 \\

            \verb[, så Car[   & \verb[, så Car[   & \verb[, så Car[      &
            \verb', så er '   & \verb', som En'   & |5.19 - 3.94| = 1.25 \\
            \hline
            \verb[; ’S[       & \verb[; ’S[       & \verb[; ’E[          &
            \verb'; Ni'       & \verb'r He'       & |3.12 - 1.78| = 1.34 \\

            \verb[; ”t[       & \verb[; ”t[       & \verb[; ”t[          &
            \verb'; ”H'       & \verb', ”j'       & |3.44 - 2.13| = 1.31 \\

            \verb[d.’ [       & \verb[d.’ [       & \verb[d.’ [          &
            \verb'ne-V'       & \verb'20’e'       & |1.75 - 2.77| = 1.02 \\

            \verb[1\n’’[      & \verb[1]’’[       & \verb[1]’’[          &
            \verb' l2-'       & \verb'720’'       & |1.75 - 2.71| = 0.96 \\

            \verb['Det[       & \verb['Det[       & \verb['Det[          &
            \verb'ndet'       & \verb' Det'       & |2.37 - 3.25| = 0.88 \\

            \verb[f 1"[       & \verb[f 1"[       & \verb[f 1"[          &
            \verb'f 2\n'      & \verb'v og'       & |3.05 - 2.20| = 0.85 \\

            \verb[æk''[       & \verb[’’ é[       & \verb[ud;'[          &
            \verb'lv; '       & \verb',tro'       & |2.60 - 1.77| = 0.83 \\

            \verb[\n\nx\n[    & \verb[\n\nx\n[    & \verb[\n\nx\n[       &
            \verb'\n\n\n\n'   & \verb'\n\n5\n'    & |1.81 - 2.61| = 0.80 \\
            \verb[ “… [       & \verb[ “… [       & \verb[?“! [          &
            \verb'nd” '       & \verb'r,” '       & |1.75 - 2.53| = 0.78 \\

            \verb[S\n, [      & \verb[S\n, [      & \verb[O\n, [         &
            \verb'e\n, '      & \verb'ad, '       & |2.62 - 1.92| = 0.70 \\
        \end{tabular}
    \end{center}
\end{frame}

\begin{frame}[fragile]{Conclusion}
    \begin{itemize}
        \item Did not perform as well as MaCom wanted.
        \item Better results than previous work on MaCom dataset.
        \item We believe that with further work we would be able to get below
            the 10\% accusation error while catching 10-20\% of cheaters.
        \item Almost succeeded on a 50\% ghostwritten dataset with an accusation
            error of 9.9\% while catching 82.1\% of the ghostwriters.
        \item Able to give teachers feedback.
    \end{itemize}
\end{frame}

\begin{frame}[fragile]{Future Work}
    \begin{itemize}
        \item Acquire some ground truth data.
        \item Combine with other machine learning methods for increased
            performance.
        \item Determine method generalizability by applying it to assignments
            from other classes.
    \end{itemize}
\end{frame}

\end{document}
